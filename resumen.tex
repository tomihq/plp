\documentclass[10pt,a4paper]{article}
\usepackage{blindtext}
\usepackage{subcaption}
\usepackage{graphicx}
\usepackage{tikz}
\usepackage{amssymb}
\usepackage{caption}
\usepackage{amsmath}
\usepackage{circuitikz}
\usepackage{hyperref}
\usepackage{amssymb}
\usepackage{amsmath}
\usepackage{listings}

\lstset{
    inputencoding=utf8,
    extendedchars=true,
    literate={á}{{\'a}}1 {é}{{\'e}}1 {í}{{\'i}}1 {ó}{{\'o}}1 {ú}{{\'u}}1 {ñ}{{\~n}}1 {Á}{{\'A}}1 {É}{{\'E}}1 {Í}{{\'I}}1 {Ó}{{\'O}}1 {Ú}{{\'U}}1 {Ñ}{{\~N}}1
}
\input{AEDmacros}
\newcommand{\notimplies}{\;\not\!\!\!\implies}
\title{Paradigmas de Lenguajes de Programación}
\author{Tomás Agustín Hernández}
\date{}

\begin{document}
\maketitle

\begin{figure}[b]
    \centering
    \begin{tikzpicture}[remember picture,overlay]
        \node[anchor=south east, inner sep=0pt, xshift=-1cm, yshift=2cm] at (current page.south east) {
            \begin{minipage}[b]{0.5\textwidth}
                \includegraphics[width=\linewidth]{logo_uba.jpg}
                \label{fig:bottom}
            \end{minipage}
        };
    \end{tikzpicture}
\end{figure}

\newpage
\section*{Recordando Haskell}
Para ejecutar un archivo hay que instalar GHCI. Una vez instalado, nos paramos en la terminal en el directorio donde está el archivo que queremos ejecutar. \\
\begin{itemize}
    \item Cargar archivo: :l nombreArchivo
    \item Ver tipo: :type tipo 
    \item Ejecutar funcion: funcion parametro1 parametro2...
    \item Recargar archivo: :r
    \item Si necesitamos hacer cálculos para mandar un parámetro, usar paréntesis: Ej.: otherwise = n * factorial(n-1)
    \item 
\end{itemize} 
\section*{Maybe}
El Maybe se utiliza en Haskell para recibir/devolver respuestas condicionales que pueden ser de un tipo u otro. \\

Se define como $data \ Maybe \ a \ = \ Nothing \ | \ Just \ a$ \\

Ej.: $ devolverFalsoSiVerdadero \:: \ Bool \rightarrow Prelude.Maybe \ Bool $ \\

El Maybe deja la puerta abierta a un valor posible "Nothing". Entonces tenemos dos casos: Si me envian un True devuelvo False (tipo bool), caso contrario, devuelvo Nothing. 

\section*{Either}
El Either se utiliza en Haskell para poder recibir/devolver un parámetro que podría ser de un tipo u otro. \\
Se define como $ data \ Either \ a \ b \ = \ Left \ a \ | \ Right \ b $ \\

Para poder saber qué operación hacer según el tipo literalmente en código usamos (Left valor) o (Right valor). \\

Ej.: $ devolverRepresentacionIntBool \ :: \ Either \ Int \ Bool \ \rightarrow \ Int $ \\

Si es un entero, devuelvo ese mismo entero porque no hago nada. Eso lo hacemos con $Left(a) \ = \ a$, ahora, si el tipo es booleano tengo que decir explícitamente la respuesta según su valor. Es decir, $Right(False) \ = \ 0$ sino, $Right(True) \ = \ 1$.
\end{document} 
